\chapter*{Передмова}
\markboth{Передмова}{}
\addcontentsline{toc}{chapter}{Передмова}

Це не підручник з~математики, не~робочий зошит.
Це лише збірка завдань з~математики, які ми вигадували для дітей
у~початкових класах БеркоШко.
Звісно, не~всі, приблизно по 40 за кожний рік.

Можливо, колись вдасться написати й підручник з~математики,
який буде яскравим і~захопливим.
Більше схожим на якісні підручники з~іноземної мови: з~дотепними персонажами,
живими малюнками, цікавими завданнями, з~додатковою методичкою для вчителя.
Але наразі хочеться поділитися вже тим, що маємо.
Бо поки система освіти буде реформована, поки з’являться підручники
нового покоління, діти вже виростуть.

Частина цих завдань була всерйоз продумана наперед,
частина~--- вигадана на ходу під час заняття:
\begin{quote}
  \itshape
  -- Ну й про кого ви~б хотіли сьогодні задачку? \\
  -- Про зелених хробачків! \\
  -- Ну гаразд. Одного дня\ldots
  \end{quote}

Крім завдань із одним рішенням, ми свідомо іноді давали дітям завдання,
які не~мають розв’язку або мають кілька можливих відповідей.

Надихайтеся, навчайтеся з~легкістю, грайтесь у~математику,
експериментуйте й творіть. \smiley

\medskip

\emph{%
Прохання до читачів: якщо раптом знайдете помилки,
повідомте про це, будь ласка, на електронну пошту\footnote{
  E-mail: \href{mailto:alexandervpetrov@gmail.com}{alexandervpetrov@gmail.com}
}. Також будемо вдячні за відгуки.}
