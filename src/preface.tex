\chapter*{Передмова}

Це не підручник з~математики, не~робочий зошит.
Це лише збірка завдань з~математики, які ми вигадували для дітей
в~початкових класах БеркоШко.
Звісно не~всі, по 40 за кожний рік.

Можливо колись вдасться написати й підручник з~математики,
який буде яскравим і~захопливим.
Більше схожим на якісні підручники з~іноземної мови, з~дотепними персонажами,
живими малюнками, цікавими завданнями, з~додатковою методичкою для вчителя.
Але наразі хочеться поділитися вже тим, що маємо.
Бо поки система освіти буде реформована, поки з’являться підручники
нового покоління, діти вже виростуть.

Частина цих завдань була всерйоз продумана наперед,
частина~--- вигадана на хочу під час заняття:
\begin{dialogue}
    \item Ну й про кого ви~б хотіли сьогодні задачку?
    \item Про зелених хробачків!
    \item Ну гаразд. Одного дня\ldots
\end{dialogue}

Окрім завдань із одним рішенням, ми свідомо іноді давали дітям завдання,
які не~мають розв’язку, або мають кілька можливих відповідей.

Надихайтесь, навчайтесь з~легкістю, грайтесь в~математику,
експериментуйте й~творіть. \smiley

\medskip
\medskip

\emph{\small
Оскільки завдання публікуються вперше, то прохання до читачів,
якщо раптом будуть знайдені помилки,
повідомити про це на електронну пошту\footnote{
    E-mail: \href{mailto:many.pure@gmail.com}{many.pure@gmail.com}
}. Також будемо вдячні за відгуки.
}%
\smiley
