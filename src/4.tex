% !TEX root = ../main.tex
\chapter{Четвертий клас}

\problem
Розділи торти порівну на всіх запрошених:
\begin{enumerate}
    \item 6~тортів на 48~гостей;
    \item 8~тортів на 48~гостей;
    \item 9~тортів на 108~гостей;
    \item 15~тортів на 120~гостей.
\end{enumerate}

\begin{figure}[h]
    \centering
    \includegraphics[width=0.3\textwidth]{4-01-1}
\end{figure}


\problem
У~кожного з~9 учнів БеркоШко є: зошити в~клітинку~--- з~математики,
з~живого світу і~два зошити з~англійської, зошити в~лінійку~---
з~російської та української мови, а~також альбоми з~музики і~дослідів.
Скільки зошитів у~всіх учнів БеркоШко разом?


\problem
Валя замовила футболки для Варі, Макса, Росави~--- бо їм не~дісталося навесні,
а~також для Федька, Машки і~Сашка, бо вони свої за літо зносили або загубили :).
За всі футболки разом вона заплатила 696~грн.
Скільки коштує друк однієї футболки? 


\problem
Бабуся в’язала шкарпетки бійцям добровольчого батальйону.
Спершу плела з~синіх ниток, а~коли сині закінчилися, почала плести коричневі.
За два тижні вона встигла сплести 14~пар шкарпеток.
\begin{enumerate}
    \item З~якою швидкістю бабуся в’яже шкарпетки? 
    \item Скільки штук шкарпеток синього кольору сплела бабуся,
    якщо вона помітила, що коричневих вийшла рівно четвертинка.
\end{enumerate}


\problem
Визнач відстані:

\begin{figure}[h]
    \centering
    \includegraphics[width=0.6\textwidth]{4-05-1}
\end{figure}

\begin{tabular}{|c|c|c|}
    \hline
    Відрізок & Приблизно, см & Точно, см \\ \hline
    BC & & \\ \hline
    AE & & \\ \hline
    ED & & \\ \hline
    AF & & \\ \hline
    EF & & \\ \hline
    FD & & \\ \hline
\end{tabular}


\problem
% TODO: meaning of the problem?
Запиши скорочену умову~--- лише те, що необхідне для розв’язання,
зазнач одиниці вимірювання величин у~дужках, пояснення до кожної дії
і~відповідь.

Першою країною, яка почала досліджувати космос, був Радянський союз (СРСР),
на~той час Україна була в~його складі. Потім активно долучилися США.
Зараз космічні дослідження переважно міжнародні.

Трохи історії перших років космонавтики:
У~1957~році в~космос запустили перший супутник «Спутник-1» (СРСР).
Перша тварина у~космічному просторі~--- собака Лайка на «Спутник-2»
у~1957 році (СРСР).
Першим космонавтом став Юрій Гагарін у~1961~році (СРСР).
Вперше до Місяця долетів корабель «Луна-2» (без людей) у~1959 році,
а~в~1970~році на Місяць доставили перший автоматичний планетохід —
«Луноход-1» (СРСР).
Вперше у~відкритий космос вийшов Олексій Леонов у~1965 році (СРСР).
Першою людиною, що висадилася на Місяць став Ніл Армстронг у~1969~році (США).
Далі були польоти до інших планет і~орбітальні станції,
кому цікаво~--- можете почитати детальніше.

Перший український космонавт часів незалежності України~--- Леонід Каденюк.
У~1997~році він літав у~космос у~складі міжнародної команди корабля «Колумбія».
Його завданням було дослідження впливу невагомості на розвиток рослин:
ріпи, сої та моху. Політ тривав 16~днів. У~нього було по 12~рослин кожного виду.
Щодня він обстежував рослини і~результати досліджень записував у~протокол,
про кожну рослинку~--- на 1~сторінку. Наприкінці польоту Леонід Каденюк
написав звіт про вплив невагомості на рослини, який мав 3~розділи.
Перший розділ~--- вступ~--- мав 37~сторінок. Другий розділ містив всі
протоколи досліджень. А~в~третьому розділі були описані висновки~---
на 16~сторінках.
(Насправді, я не~знаю, скільки саме сторінок було у~звіті Леоніда Каденюка,
але сам космонавт, його політ і~дослідження рослин~--- точно були \smiley)


\problem
Виріши рівняння:
\begin{multicols}{2}
    \begin{enumerate}
        \item $4096 = 6194 - x$
        \item $678 - 3x = 597$
        \item $87 - x = x + 37$
        \item $2x + 66 = 117 - x$
        \item $831 - 3x = x + 391$
    \end{enumerate}
\end{multicols}


\problem
\problemname{Задача про авантюриста Джека}

% TODO: meaning of the problem?

\begin{quote}
\itshape
    -- А~хто такий авантюрист? \\
    -- Слово походить від французького «aventure»~--- «пригода».
    Але це не будь-яка пригода.
    В~українській і~російській мові слово набуло певного відтінку\ldots
    Зараз зрозумієте.
\end{quote}

Жив собі Джек. І~якось йому спало на думку вирушити на розшуки скарбів,
але у~нього не~було ні~карти, ні~команди, нічого. Тоді він позичив
у~сусідів 20~піастрів і~вирушив у~порт, найматися на корабель.

\begin{wrapfigure}{l}{0.4\textwidth}
    \begin{center}
        \includegraphics[width=0.4\textwidth]{4-08-1}
    \end{center}
\end{wrapfigure}

Як і~всім авантюристам, на початку йому дуже щастило.
У~портовому кабаку він познайомився з капітаном, який саме шукав
матросів на корабель. Подейкували, що саме для пошуку скарбів.
Капітан спершу не хотів брати Джека, але той віддав йому всі свої
гроші і~сказав, що працюватиме просто за харчі, аби його тільки взяли.

Так корабель із капітаном і дванадцятьма матросами, одним з~яких був Джек,
вирушив на пошуки. У~капітана справді знайшлася карта закопаних скарбів,
а ще він виявився чесною людиною і~запропонував розділити знахідку по-чесному:
всім матросам однакові долі, а~капітанові~--- вдвічі більше, бо карта його.
Всі з цим погодились.

4~роки корабель шукачів нишпорив морями та океанами у~пошуках острова
скарбів і~зрештою таки знайшов. Велику скриню, в~якій виявилося
2340 золотих піастрів.

Їх розділили: капітанові~--- 2~долі, 12~матросам~--- по одній:

Ті монети, що не~змогли поділити, кинули в~море~--- на знак вдячності
за сприятливий вітер. І~вся команда вирішила пливти далі у~пошуках пригод,
але Джекові вже дуже кортіло додому. Тому він попрощався, забрав свою
частку скарбу і~зійшов у~найближчому порту. Повертатися додому йому було
через півсвіту, але пощастило домовитися з~кораблем, що вирушав у~його
рідні краї, за 10~піастрів.

Плив Джек дуже довго, майже рік, бо саме настав сезон штормів.
Втім таки дістався додому.

Першим ділом він пішов до сусіда віддати борг.

А~той розказав Джекові, що вирушаючи в~подорож, Джек забув заплатити
ренту за будинок і~всі ці 5~років банк нараховував йому борг.

Спантеличений Джек пішов у~банк, де йому виписали квитанцію на 150~піастрів,
які він мав сплатити негайно, аби не потрапити у~боргову яму.

Джек порахував гроші і~зажурився. Довелося знову позичати у~сусіда.

А~ввечері Джек сидів на ґанку свого будинку і~мріяв про те,
яким би мав бути той скарб, щоб йому не довелося знову залазити в~борги\ldots

\begin{quote}
    \itshape
    -- Так от, авантюрист може віддати 5~років свого життя
    на розшуки скарбів, лишившися по тому ні з~чим \smiley
\end{quote}


\problem
Намалюй положення стрілок за підказками:
\begin{enumerate}
    \item пів на восьму;
    \item 09:15;
    \item за чверть п’ята;
    \item 16:55;
    \item двадцять хвилин по дев’ятій;
    \item двадцять хвилин на дев’яту;
    \item перша;
    \item за 25 хвилин десята.
\end{enumerate}

\begin{figure}[h]
    \centering
    \includegraphics[width=0.3\textwidth]{4-09-1}
\end{figure}


\problem
На Берківці потрібно було замовити дрова.
У~кожну з~5~хаток П'ятихаток і~ще у~Рогатий дім потрібно по 4~куб.~м дрів,
а~в~школу~--- 6~куб.~м.
У~самоскид вміщається 6~куб.~м, які коштують 2800~грн.

Скільки машин дрів необхідно замовити? Скільки це коштуватиме? 
Як розподілити дрова між усіма хатками порівну?

\begin{figure}[h]
    \centering
    \includegraphics[width=0.8\textwidth]{4-10-1}
\end{figure}

Коли привезли першу машину, самоскид скинув дрова на купу і~5~людей
складало дрова в дровітниці. Їм знадобилося 3~години.
Скільки часу знадобиться 3~людям, щоб розкласти таку саму кількість дрів?
А~якщо комусь доведеться працювати насамоті, скільки часу він складатиме дрова?


\problem
Дізнайся довжини мостів в~Києві: Московського, Патона,
Паркового («Мосту закоханих») в Маріїнському парку, Залізничного мосту.
Визнач найдовший, найкоротший.
Якщо довжина тролейбуса 18~м, то скільки таких тролейбусів
вміститься на кожному мосту, впритул один за одним?


\problem
Тома прочитала цікаву книжку за 15~днів. У~книжці було 315~сторінок.
Потім цю книжку попросила почитати Іванка. Їй книга так сподобалася,
що вона читала кожного дня на 7~сторінок більше, ніж Тома.
Через скільки днів Іванка зможе повернути Томі книжку?


\problem
Виріши нерівності:
\begin{multicols}{2}
    \begin{enumerate}
        \item $9000 : a > 450$
        \item $11 \cdot x > 1870$
        \item $4 \cdot y - 8 \leqslant 100$
        \item $5 \cdot (k + 17) \geqslant 175$
    \end{enumerate}
\end{multicols}


\problem
Підрахуй, скільки хлібу з’їдає твоя родина за тиждень.
Якого розміру поле мала обробити родина, щоб мати хліб на весь рік?
(У середньому родина з~чотирьох осіб~--- тато, мама і~двоє діток~---
з’їдає 4~хлібини на тиждень. Вважаємо, що 1~хлібина важить 1~кг,
а~борошна на неї потрібно 400~г. Вихід борошна із зерна~--- 4/5 його ваги.
А~на одному гектарі вирощують від 10 до 150 центнерів пшениці,
залежно від врожайності ґрунтів і~сприятливості умов.
Приймаємо середній показник по Україні за часів СРСР~---
40~центнерів з~1~гектара.)


\problem
Том Сойєр сам міг би пофарбувати паркан за 6~годин,
а~Гекльберрі Фінн~--- за 3~години. Але вони почали фарбувати його вдвох,
одночасно, з~протилежний кінців, назустріч один одному.
Через який час вони зустрінуться, пофарбувавши весь паркан,
довжина якого 120~м?


\problem
Зустрілись у~прерії два ковбої Том і~Джері. Постояли, побалакали та й
роз’їхалися кожний на своє ранчо. Том поїхав зі швидкістю $v$ на схід,
а~Джері, зі швидкістю на $x$ більшою~--- на захід. Їхали до своїх ранчо
вони однаковий час $t$. Яка відстань між ранчо Тома і~Джері?

А~тепер обчисли відстань, якщо швидкість Тома 47~м/хв,
у~Джері~--- на 4~м/хв більше, а~роз’їжджалися вони 2~години?


\problem
Відстань між містом Бурмосиків і~містом Викрутасиків~--- 204~км.
Викрутасики добігають у~гості до Бурмосиків за 6~годин.
За скільки годин доходять у~гості Бурмосики,
якщо їхня швидкість на 22~км/год менша, ніж швидкість Викрутасиків?


\problem
\problemname{Незавершена мандрівка}
Троє хлопців з~містечка Радомишль, що на річці Тетерів, зробили пліт.
І~зібралися вирушити на ньому в~мандри до моря. Запаслися харчами,
теплим одягом, картою, різним інструментом і о~9~ранку відпливли вниз
за течією. Вже опівдні вони пропливли повз село Березці, в якому жила
бабуся одного з~хлопців, а~він знав, що до того села відстань 6~км.
Отже вони могли дізнатися, з~якою швидкістю вони сплавляються,
і~розрахувати, коли дістануться до моря.

\begin{wrapfigure}{r}{0.4\textwidth}
    \begin{center}
        \includegraphics[width=0.4\textwidth]{4-18-1}
    \end{center}
\end{wrapfigure}

Але не все так сталось, як гадалось. Цього дня тато одного з~хлопців
о~9:00 відплив на своєму катері вверх по течії. Катер був доволі швидкий,
минулого літа на озері він розганявся до 14~км/год.

І~ось, коли відстань між ними (катером і~плотом) становила вже 84~км,
тато зрозумів, що хлопці попливли в~мандри і~до вечері додому не встигнуть.
Розвернувся і~поплив їх наздоганяти.

Скільки часу минуло з~відчалювання у~Радомишлі до того, як тато
розвернув човна і~поплив навздогін хлопцям?
Скільки часу він їх наздоганятиме?
Яку відстань встигнуть пропливти хлопці на плоті, поки тато їх не перехопить?


\problem
Прямокутний лист фанери довжиною 5~дм і~шириною 3~дм розпиляли на квадратики
зі стороною 1~см.
Скільки таких квадратиків вийшло?


\problem
Збираючися в~похід, Макс купив собі новий тент і~для міцності вирішив
обшити його тасьмою. Ширина тенту~--- 6~м, а довжина~--- на 4~м більша.
Скільки метрів тасьми йому знадобиться для цього?


\problem
Вовк пробіг певну відстань за якийсь час.
Лось чверть цього шляху пробіг за час, вдвічі менший.
У скільки разів швидкість лося менша за швидкість вовка?


\problem
За прогнозом погоди від 1~січня 2015~року температура повітря вдень
у~Києві складатиме:

\begin{multicols}{2}
    7~січня~--- 0~градусів

    8~січня~--- 0~градусів

    9~січня~--- 0~градусів

    10~січня~--- 2~градуси

    11~січня~--- 3~градуси

    12~січня~--- 4~градуси

    13~січня~--- 3~градуси

    14~січня~--- 2~градуси

    15~січня~--- 2~градуси

    16~січня~--- 4~градуси
\end{multicols}

Яку середню температуру повітря очікують у~Києві з~7 до 16~січня?


\problem
Виріши рівняння:
\begin{multicols}{2}
    \begin{enumerate}
        \item $x : 100 = 0$
        \item $x + 100 = 100$
        \item $x \cdot 100 = 100\,000$
        \item $x - 100 = 100\,000$
    \end{enumerate}
\end{multicols}


\problem
Обчисли:
\begin{enumerate}
    \item $1530 : (12 \cdot 6 - 38) \cdot 15$
    \item $4840 + 12\,903 - 80\,125 : 5$
    \item $(420 \cdot 20 - 210 \cdot 30) : 100 \cdot (600 - 591)$
    \item $(500 \cdot 10 - 500 : 10) \cdot 6 : (1000 - 997)$
\end{enumerate}


\problem
Щомісяця батьки дають хлопчикові 100~грн, з~них 65~грн він витрачає
на проїзд і~смаколики, а~решту відкладає на придбання моделей літачків.
Через який час він назбирає достатньо грошей, щоб придбати 50~моделей літачків,
кожний з~яких коштує 28~грн?


\problem
Жіночка продає в~метро календарі.
Закупає у~видавництві 12~календарів за 50~грн, а~продає кожний по 5~грн.
Скільки вона заробить, якщо продасть 30~календарів?


\problem
Черепашка з~дитинства мріяла подорожувати. І~ось одного сонячного ранку
вона поклала в~наплечник GPS-навігатор, щоб не заблукати,
і~вирушила в~мандрівку.

Першого дня вона доповзла до лісу і~влаштувалася на ночівлю під крислатим
дубом. Перед сном черепашка визначила за навігатором, що цього дня
проповзла рівно 2645~м.

Другого дня вона мандрувала лісом, шаруділа листям, милувалася сонячними
променями, які пробивалися крізь гілля, ласувала ожиною і~суницями.
І~ввечері знову перевірила за навігатором, яку~ж відстань подолала.
Виявилося, що за другий день вона проповзла рівно стільки~ж,
скільки й за перший.

Третього дня їй вдалося доповзти до гарного лісового озера,
вона влаштувалася на крутому беріжку і~спробувала визначити,
яку відстань вона проповзла, але в~навігаторі щось заглючило
і~він не~показував відстані.

Черепашка засмутилась і~почала натискати на всі кнопочки меню врізнобій,
аж~раптом навігатор видав повідомлення, що за третій день вона проповзла
на 1740~м менше, ніж за попередні два дні разом.

То як далеко від дому відповзла черепашка?


\problem
Підприємець придбав 18~дюжин стільців по 3000~грн за дюжину.
За перевезення заплатив 800~грн.
Відтак кожний стілець продав за 300~грн.
Скільки він заробив?


\problem
У~коробці 80~сірників, коштує вона 1~грн.
Скільки коштують 2000~сірників?


\problem
Перед тим, як їхати з~БеркоШко на музику, Клим і~Макс сіли перекусити курагою.
Макс з’їв вдвічі більше за Клима, а~Клим на 6~куражинок менше за Макса.
Скільки кураги з’їв Клим?


\problem
На яке число треба поділити 22849, щоб вийшло 36 і~ще 25 в~остачі?


\problem
По дорозі до школи Клим відшукує воронячі гнізда. Вже нарахував 21~гніздо.
Навесні в~кожному з~них з’явиться в~середньому по 3~вороненятка.
Цікаво, скільки вороненят народиться тут за 5~років?


\problem
Чи можна намалювати хатку, на відриваючи олівця від паперу і~на проводячи
жодну лінію двічі?
З~якої точки варто починати?

\begin{figure}[h]
    \centering
    \includegraphics[width=0.3\textwidth]{4-33-1}
\end{figure}


\problem
На Великому Пагорбі на сторожі сиділи двоє індіанців:
Орлине Око з~племені ірокезів і~Вертлявий Тхір з~племені делаварів.

Раптом вони помітили на узліссі блідошкірих, які шикувалися для нападу.
Кожний з~індіанців побіг попередити своє плем’я.
Табори обох племен розташовані на однаковій відстані від Великого Пагорба~---
5~кілометрів, але делавари дізналися про напад блідолицих через 20~хвилин,
а~ірокези~--- через 30~хвилин.

На скільки Вертлявий Тхір бігає швидше за Орлине Око?


\problem
Збільши у~68~разів частку суми чисел 4893 і~1527 та числа 30.


\problem
Два потяги одночасно виїхали назустріч одне одному.
Швидкість одного 80~км/год, а~другого~--- 7/8 від швидкості першого потяга.
Через 2~години після початку руху їм залишалося проїхати
1/3 початкової відстані.
Скільки кілометрів становила відстань між потягами на початку?


\problem
Який середній вік людей, які перебувають у~кімнаті просто зараз?


\problem
Одного дня беркошколярики прийшли до школи як завжди о~9:15.
Знайшли на поличці пісочний годинник на 15 хвилин і~почали відміряти ним час.
Щойно пісок пересипався з~однієї чаші годинника в~іншу,
його знову перевертали, навіть під час занять,
навіть під час великої перерви.
Впродовж дня діти встигли перевернути годинник 27~разів.
І~коли пісок пересипався востаннє, всі пішли додому.
О~котрій годині всі пішли додому того дня?


\problem
% TODO: понавидувати?
Для розписування писанок перед Великоднем Женя попросила всіх понавидувати
вдома яєць і~принести до БеркоШко. Коли всі діти поскладали принесені яйця
в~лоточки, вдалося заповнити 4~лотки на десяток яєць і~2~--- на~18.
Ще один лоток на 15~яєць принесла з~собою Женя.
Діти старанно робили писанки, топили віск, обережно малювали писачками,
але декілька яєць все~ж таки висковзнули з~рук і~розбилися.
Коли діти закінчили заняття писанкарством, то нарахували 87~готових писанок.
То скільки яєць розбилося?


\problem
Якось Тома, Юля, Клим, Федько, Іванка, Варя, Маша і~Макс гралися
в~індіанців і~випадково знайшли поза школою моток ліски, довжиною 7~м 44~см.
І~вирішили зробити з~неї тятиву для своїх луків.
Як розділити ліску на всіх порівну і~якої довжини тятива вийде у~кожного?
